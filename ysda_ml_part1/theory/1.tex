\documentclass{article}

\usepackage[utf8]{inputenc}
\usepackage{amsmath, amssymb}
\usepackage[english,russian]{babel}
\usepackage[T1]{fontenc}
\usepackage[left=1.5cm,right=1.5cm,top=2cm,bottom=2cm,bindingoffset=0cm]{geometry}
\usepackage{tikz}
\usepackage{pgfplots}

\begin{document}
    \textbf {Задача 1.}
    В случае, когда обучающая выборка содержит очень мало объектов, k-fold с небольшим k не подойдет, так как при нем обучение будет вестись на части и без того очень маленькой выборки.
    В этом случае лучше использовать Leave-one-out-CV, так как при обучении выкидывается всего один объект и проверка проводится на нем.
    В итоге каждый объект выборки поучаствует в контроле ровно один раз.
    То есть при обучении мы теряем всего лишь один элемент выборки, а так как выборка маленькая, то обучение N-1 раз не потребует много времени.

    В случае же, когда обучающая выборка очень велика, Leave-one-out не подойдет, так как это потребует очень много времени на обучение N-1 раз.
    В этом случае лучше использовать k-fold.
    При этом обучение будет производиться k раз (k мало), а так как объектов в выборке очень много, качество обучения не сильно пострадает при обучении на $\frac{k-1}{k}$ объектов.

    \textbf {Задача 2.}
    $\rho_p(x,y)=(\sum_{i=1}^n|x_i-y_i|^p)^{\frac{1}{p}}$

    При $n=2$, $\rho_p(x,y)=(|x_1-y_1|^p+|x_2-y_2|^p)^{\frac{1}{p}}$, $f_p(x)=\rho_p(x,0)=(|x_1|^p+|x_2|^p)^{\frac{1}{p}}$

    Евклидова метрика: $f_2(x)=\rho_2(x,0)=\sqrt{x_1^2+x_2^2}$

    Манхэттенское расстояние: $f_1(x)=|x_1|+|x_2|$

    Метрика Чебышева: $f_\infty(x)=\lim_{n\to\infty}(x_1^n+x_2^n)^{\frac{1}{n}}$

    Покажем, что $\lim_{n\to\infty}(x_1^n+x_2^n)^{\frac{1}{n}}=\max\{x_1,x_2\}$

    БОО, пусть $\max\{x_1,x_2\}=x_1$, тогда
    $\lim_{n\to\infty}(x_1^n+x_2^n)^{\frac{1}{n}}\geq\lim_{n\to\infty}(x_1^n)^{\frac{1}{n}}=\lim_{n\to\infty}x_1=x_1=\max\{x_1,x_2\}$

    С другой стороны, $\lim_{n\to\infty}(x_1^n+x_2^n)^{\frac{1}{n}}\leq\lim_{n\to\infty}(x_1^n+x_1^n)^{\frac{1}{n}}=\lim_{n\to\infty}(2x_1^n)^{\frac{1}{n}}=x_1\lim_{n\to\infty}2^{\frac{1}{n}}=x_1=\max\{x_1,x_2\}$

    Таким образом, получатся следующие графики:

    \begin{tikzpicture}[scale=1.5]
        \draw[->,thick] (-1,0)--(1,0) node[right]{$x_1$};
        \draw[->,thick] (0,-1)--(0,1) node[above]{$x_2$};

        \draw (0,0) circle [radius=0.7];

        \node[align=center,font=\bfseries, yshift=2em] (title)
            at (current bounding box.north)
            {Евклидова метрика};
    \end{tikzpicture}
    \begin{tikzpicture}[scale=1.5]
        \draw[->,thick] (-1,0)--(1,0) node[right]{$x_1$};
        \draw[->,thick] (0,-1)--(0,1) node[above]{$x_2$};

        \draw (0,0.7)--(0.7,0);
        \draw (0.7,0)--(0,-0.7);
        \draw (0,-0.7)--(-0.7,0);
        \draw (-0.7,0)--(0,0.7);

        \node[align=center,font=\bfseries, yshift=2em] (title)
            at (current bounding box.north)
            {Манхэттенское расстояние};
    \end{tikzpicture}
    \begin{tikzpicture}[scale=1.5]
        \draw[->,thick] (-1,0)--(1,0) node[right]{$x_1$};
        \draw[->,thick] (0,-1)--(0,1) node[above]{$x_2$};

        \draw (-0.7,0.7)--(0.7,0.7);
        \draw (0.7,0.7)--(0.7,-0.7);
        \draw (0.7,-0.7)--(-0.7,-0.7);
        \draw (-0.7,-0.7)--(-0.7,0.7);

        \node[align=center,font=\bfseries, yshift=2em] (title)
            at (current bounding box.north)
            {Метрика Чебышева};
    \end{tikzpicture}

    \textbf {Задача 3.}

    Пусть $\xi$ - минимальное расстояние от центра шара до точки, тогда
    $F_\xi(x)=P(\xi\leq x)=P($в шаре радиуса x есть хотя бы одна точка$)=1-P($все точки вне шара$)$.
    Вероятность того, что точка находится вне шара радиуса $x$ можно получить если вычесть из объема единичного шара объем шара радиуса $x$ и поделить это на объем единичного шара, то есть
    $P=\frac{V_1 -V_x}{V_1}=1-\frac{V_x}{V_1}$.
    Так как точки распологаются независимо, вероятность того, что все точки находятся вне шара будет равна $(1-\frac{V_x}{V_1})^l$.

    Объем n-мерного шара радиуса x равен $\frac{\pi^{\frac{n}{2}}}{\Gamma(\frac{n}{2}+1)}x^n$.
    Подставим в формулу:

    $P($в шаре радиуса x есть хотя бы одна точка$)=1-(1-x^n)^l$

    Так как x - медиана, то $F_\xi(x)=0.5$, $1-(1-x^n)^l =0.5$, $(1-x^n)^l =\frac{1}{2}$, $1-x^n =2^{-\frac{1}{l}}$, $x=\sqrt[n]{1-2^{-\frac{1}{l}}}$.

    При $l=500$, $n=10$, $x\approx 0.5178$

    \begin{tikzpicture}
        \begin{axis}[grid=both,
            xmax=100,ymax=1,
            axis lines=middle,
            restrict y to domain=0:1,
            enlargelimits,
            xlabel={$n$},
            ylabel={$x$}]
            \addplot[green,domain=0:100]  {(1-2^(-1/500))^(1/x)};
        \end{axis}
    \end{tikzpicture}

    Видно, что при увеличении размерности $x$ быстро растет.



    Возьмем теперь значение x=0.5 (будем считать, что мы побороли проклятие размерности при таком значении).
    Тогда $l=-\frac{1}{\log_2 (1-2^{-n})}$.

    Возьмем n=100.
    Для такой размерности чтобы побороть проклятие размерности нужно $9.7\cdot 10^{13}$ точек, то есть очень много.
    Таким образом, данная формула наглядно демонстрирует проклятие размерности.

    \textbf {Задача 4.}

    \begin{tikzpicture}[scale=1.5]
        \draw [<->,thick] (0,1.5) node (yaxis) [above] {$y$}
            |- (1.5,0) node (xaxis) [right] {$x$};

        \draw[dashed] (0.2,1.5)--(0.2,0) node [below] {$0.1$};
        \draw[dashed] (1,1.5)--(1,0) node [below] {$0.5$};

        \draw[dashed] (1,0.5)--(0,0.5) node [left] {$y_2$};
        \draw[dashed] (0.2,1)--(0,1) node [left] {$y_1$};

        \draw (0,0)--(1,0.5);
        \draw (0,0)--(0.2,1);
    \end{tikzpicture}

    $y_1,y_2\in U([0,1])$.
    Нужно найти вероятность того, что $0.5^2+y_2^2<0.1^2+y_1^2$, то есть $P(y_1^2-y_2^2>0.24)$

    \begin{tikzpicture}[scale=1.5]
        \draw [<->,thick] (0,1.5) node (yaxis) [above] {$y_2$}
            |- (1.5,0) node (xaxis) [right] {$y_1$};

        \fill[gray,domain=0.49:1,smooth,variable=\x] (0.49,0)--plot (\x,{sqrt(\x^2-0.24)})--(1,0)--cycle;
        \draw (1,0)--(1,1);
        \draw (0,1)--(1,1);
    \end{tikzpicture}

    Нужно найти площадь под графиком $y_2=\sqrt{y_1^2-0.24}$, для этого достаточно найти
    $\int_{\sqrt{0.24}}^1\sqrt{x^2-0.24}dx\approx 0.275$

    \textbf {Задача 5.}

    В данном случае число K - гиперпараметр метода KNN\@.
    Параметры KNN - координаты объектов из обучающей выборки.
    В результате у линейной модели на рисунке 1 всего 2 параметра, а у KNN намного больше.

    \textbf {Задача 6.}

    Вероятность получить на объекте $x_i$ из выборки ответ 1 равна $p(x_i)^{y_i}$, вероятность получить 0 - $(1-p(x_i))^{1-y_i}$.
    Таким образом, функция правдоподобности примет вид

    $L=\prod_{i=1}^l p(x_i)^{y_i}(1-p(x_i))^{1-y_i}$, прологарифмируем:

    $logL=\sum_{i=1}^l (y_i \log{p(x_i)}+(1-y_i)\log(1-p(x_i)))$, что равно формуле для LogLoss со знаком минус.

    \textbf {Задача 7.}

    1. $(\frac{1}{N}\sum_{i=1}^N(y_i-y)^2)'=(\frac{1}{N}(\sum_{i=1}^N y_i^2-2\sum y_i y+\sum y^2))'=\frac{1}{N}(-2\sum y_i+2Ny)$
    $=-\frac{2}{N}\sum y_i+2y=0$,
    $y=\frac{1}{N}\sum y_i$.
    То есть нужно взять выборочное среднее.
    При этом, если мы возьмем величину меньше выборочного среднего, производная будет отрицательной, а если возьмем больше - положительной, следовательно, это точка минимума.

    2. $(\frac{1}{N}\sum_{i=1}^N |y_i-y|)'=\frac{1}{N}\sum \mathrm{sign}(y-y_i)=0$.
    Производная будет равна 0 если взять медиану выборки.
    При этом если взять величину меньше медианы, производная будет отрицательной (в сумме знаков будет больше значений -1 чем 1), а если больше - положительной, следовательно, это точка минимума.

    3. $(-\frac{1}{N}\sum_{i=1}^N (y_i\log{y}+(1-y_i)\log(1-y)))'=-\frac{1}{N}\sum y_i \frac{1}{y}-(1-y_i)\frac{1}{1-y}=0$

    $\sum \frac{y_i(1-y)-(1-y_i)y}{y(1-y)}=0$, $\sum\frac{y_i -y}{y(1-y)}=\sum(\frac{y_i}{y(1-y)}-\frac{1}{1-y})=0$,
    $\frac{N}{1-y}=\sum\frac{y_i}{y(1-y)}$, $y=\frac{1}{N}\sum y_i$.
    То есть нужно взять выборочное среднее.
    При этом производная равна $-\frac{1}{y(1-y)}(\frac{1}{N}\sum y_i -y)$.
    Так как $y_i$ равны 0 или 1, $y$ будет лежать межну 0 и 1, то есть $-\frac{1}{y(1-y)}$ всегда будет отрицательным.
    Разность в скобках будет положительной если y меньше выборочного среднего и отрицательной если меньше.
    То есть в данной точке действительно достигается минимум функции.

    \textbf {Задача 8.}

    1. Выборочная дисперсия для выборки размера $l$ равна $\frac{1}{l}\sum_{i=1}^l (y_i-\overline{y})^2$

    При $L(y_i, \overline{y})=(y_i-\overline{y})^2$, $\Phi(U)=\frac{1}{l}\sum_{i=1}^l (y_i-\overline{y})^2$,
    то есть выборочной дисперсии.

    \textbf {Задача 9.}

    1. $E\frac{1}{N}\sum_{i=1}^N [y_i \neq a(x_i)]=\frac{1}{N}\sum_{i=1}^N E[y_i \neq a(x_i)]=\frac{1}{N}\sum_{i=1}^N (1-p_{my_i})$

    Разобъем по классам:
    $=\sum_{k=1}^K \frac{1}{N}\sum_{i=1}^N [y_i=k](1-p_{mk})=[\frac{1}{N}\sum_{i=1}^N [y_i=k]=p_{mk}]$
    $=\sum_{k=1}^K p_{mk}(1-p_{mk})$, что равно индексу Джини.

\end{document}